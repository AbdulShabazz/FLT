\chapter{Miniproject: Haar Characters}\label{Haar_char_project}

\section{The goal}

The goal of this miniproject is to develop the theory (i.e., the basic API) of Haar characters.
``Haar character'' is a name I've made up to describe a certain character of the units of a locally
compact topological ring. The main result we need here is that if $B$ is a finite-dimensional
algebra over a number field~$K$, then $B^\times$ is in the kernel of the Haar character
of $B\otimes_K\A_K$, where $\A_K$ is the ring of adeles of~$K$. Most if not all of this
should probably be in mathlib.

KMB would like to heartily thank S\'ebastien Gou\"ezel for the help he gave during the preparation
of this material.

\section{Initial definitions}

\subsection{Scaling Haar measure on a group}

Let $A$ be a locally compact topological additive abelian group. There's then a regular additive
Haar measure $\mu$ on $A$, unique up to a positive scalar factor. If $\phi:(A,+)\cong(A,+)$ is a
homeomorphism and an additive automorphism of $A$, then we can push forward $\mu$
along $\phi$ to get a second measure $\phi_*\mu$ on $A$, with the property that
$(\phi_*\mu)(X)=\mu(\phi^{-1}X)$ for any Borel subset $X$ of $A$.

Now $\phi_*\mu$ is a translation-invariant and regular measure,
and hence also a Haar measure on $A.$ It must thus differ from
$\mu$ by a positive scalar factor, which we call $d_A(\phi)$.
There is a choice of normalization here between $d_A(\phi)$
and $d_A(\phi)^{-1}$, so let us be more precise.

\begin{definition}
  \label{MeasureTheory.addEquivAddHaarChar}
  \lean{MeasureTheory.addEquivAddHaarChar}
  \leanok
  If $A$ is a locally compact topological additive abelian group,
  if $\mu$ is a regular Haar measure on $A$, and if $\phi:A\to A$ is an
  additive homeomorphism, then we let $d_A(\phi)$ denote the unique positive
  real number such that $\mu(X)=d_A(\phi)(\phi_*\mu)(X)$ for any Borel set~$X$.
\end{definition}

To give an example, if $\phi$ is multiplication by $2$ on the real numbers,
if $X=[0,1]$, and if $\mu$ is Lebesgue measure on the Borel subsets of $\R$,
we have that $\phi_*\mu(X)=\mu(\phi^{-1}(X))=\mu([0,1/2])=1/2$,
so $1=d_A(\phi)/2$ meaning that $d_A(\phi)=2$. Similarly if $\phi$ is multiplication
by $-2$ and $X=[0,1]$ then $\phi^{-1}(X)=[-1/2,0]$ which again has measure $1/2$,
so $d_A(\phi)$ is 2 again.

Strictly speaking our definition of $d_A(\phi)$ depends on the choice of regular Haar
measure $\mu$. Note that {\tt mathlib} offers a fixed Borel regular Haar measure
{\tt MeasureTheory.Measure.haar} on any locally compact topological group
and the actual definition of $d_A$ in the code uses this definition.
Note also that the code defines everything for multiplicative groups
and uses {\tt @[to\_additive]} to deduce the corresponding results
for additive groups.

Here are some basic results about this construction. In all of them,
$A$ is a locally compact topological group and $\phi:A\to A$ is
a group isomorphism and a homeomorphism. The first lemma shows that
the definition of $d_A(\phi)$ is indeed independent of the choice of Haar measure.

\begin{lemma}
  \label{MeasureTheory.addEquivAddHaarChar_eq}
  \lean{MeasureTheory.addEquivAddHaarChar_eq}
  \uses{MeasureTheory.addEquivAddHaarChar}
  \discussion{508}
  \leanok
  $d_A(\phi)$ is independent of choice of regular Haar measure.
\end{lemma}
\begin{proof}
  \leanok
  If $\mu'$ is a second choice then $\mu'=\lambda\mu$ for some
  positive real $\lambda$, and the $\lambda$s on each side of
  $\mu'(X)=d_A(\phi)(\phi_*\mu')(X)$ cancel.
\end{proof}

\begin{lemma}
  \lean{MeasureTheory.addEquivAddHaarChar_map}
  \label{MeasureTheory.addEquivAddHaarChar_map}
  \uses{MeasureTheory.addEquivAddHaarChar_eq}
  \leanok
  If $\mu$ is any regular Haar measure on $A$ then
  $d_A(\phi)(\phi_*\mu) = \mu.$
\end{lemma}
\begin{proof}
  \leanok
  This is a restatement of the previous result.
\end{proof}

We can of course also pull a Haar measure $\mu$ back along a homeomorphism $\phi$,
giving a measure $\phi^*\mu$ such that $\phi^*\mu(X)=\mu(\phi(X))$.

\begin{corollary}
  \lean{MeasureTheory.addEquivAddHaarChar_comap}
  \label{MeasureTheory.addEquivAddHaarChar_comap}
  \uses{MeasureTheory.addEquivAddHaarChar_map}
  \leanok
  If $\mu$ is any regular Haar measure on $A$ then
  $d_A(\phi)\mu = \phi^*\mu.$
\end{corollary}
\begin{proof}
  \leanok
  This follows from lemma~\ref{MeasureTheory.addEquivAddHaarChar_map}
  applied to the regular Haar measure $\phi^*\mu$ and the fact that $\phi_*\phi^*\mu=\mu$.
\end{proof}

Now let $\mu$ be any regular additive Haar measure on $A$.

\begin{lemma}
  \label{MeasureTheory.addEquivAddHaarChar_smul_preimage}
  \lean{MeasureTheory.addEquivAddHaarChar_smul_preimage}
  \uses{MeasureTheory.addEquivAddHaarChar}
  \discussion{509}
  \leanok
  If $X$ is a Borel set then $\mu(X)=d_A(\phi)\mu(\phi^{-1}X)$.
\end{lemma}
\begin{proof}
  \leanok
  \uses{MeasureTheory.addEquivAddHaarChar_map}
  This follows immediately from lemma~\ref{MeasureTheory.addEquivAddHaarChar_map}
  and the definition of the pushforward of a measure.
\end{proof}

\begin{lemma}
  \label{MeasureTheory.addEquivAddHaarChar_smul_integral_map}
  \lean{MeasureTheory.addEquivAddHaarChar_smul_integral_map}
  \uses{MeasureTheory.addEquivAddHaarChar}
  \discussion{510}
  \leanok
  If $f:A\to\R$ is a Borel measurable function then
  $d_A(\phi)\int f(x)d\phi_*\mu(x)=\int f(x)d\mu(x)$.
\end{lemma}
\begin{proof}
  \leanok
  \uses{MeasureTheory.addEquivAddHaarChar_map}
  This also follows immediately from lemma~\ref{MeasureTheory.addEquivAddHaarChar_map}.
\end{proof}

We also have the following variant:

\begin{lemma}
  \label{MeasureTheory.addEquivAddHaarChar_smul_integral_comap}
  \lean{MeasureTheory.addEquivAddHaarChar_smul_integral_comap}
  \uses{MeasureTheory.addEquivAddHaarChar_comap}
  \leanok
  If $f:A\to\R$ is a Borel measurable function then
  $d_A(\phi)\int f(x)d\mu(x)=\int f(x)d\phi^*\mu(x)$.
\end{lemma}
\begin{proof}
  \leanok
  This is immediate from corollary~\ref{MeasureTheory.addEquivAddHaarChar_comap}.
\end{proof}

Note that as a consequence of lemma~\ref{MeasureTheory.addEquivAddHaarChar_smul_preimage},
if $X$ is a Borel subset of $A$ with positive finite measure then we can read
off $d_A(\phi)$ by $d_A(\phi)=\mu(X)/\mu(\phi^{-1}(X))$, and hence also by
$d_A(\phi)=\mu(\phi(X))/\mu(X)$. A nice special case is when
$\mu(X)=1$, in which case we have $d_A(\phi)=\mu(\phi(X))$ for all $\phi$,
or $d_A(\phi)=1/\mu(\phi^{-1}(X))$.
Similarly, by lemma~\ref{MeasureTheory.addEquivAddHaarChar_smul_integral_map},
if $f$ is a measurable function with $0<\int f(x)d\mu(x)<\infty$ then
we can read off $d_A(\phi)$ by $d_A(\phi)=(\int f(x)d\mu(x))/(\int f(x)\phi_*\mu(x))$.
Note that {\tt mathlib} supplies such $f$ with the function {\tt exists\_continuous\_nonneg\_pos}.
The following are also straightforward.

\begin{lemma}
  \label{MeasureTheory.addEquivAddHaarChar_refl}
  \lean{MeasureTheory.addEquivAddHaarChar_refl}
  \uses{MeasureTheory.addEquivAddHaarChar}
  \leanok
  $d_A(id)=1.$
\end{lemma}
\begin{proof}
  \uses{MeasureTheory.addEquivAddHaarChar_eq}
  \leanok
  Clear.
\end{proof}

\begin{lemma}
  \label{MeasureTheory.addEquivAddHaarChar_trans}
  \lean{MeasureTheory.addEquivAddHaarChar_trans}
  \uses{MeasureTheory.addEquivAddHaarChar}
  \leanok
  \discussion{511}
  $d_A(\phi\circ\psi)=d_A(\phi)d_A(\psi).$
\end{lemma}
\begin{proof}
  \leanok
  \uses{MeasureTheory.addEquivAddHaarChar_smul_preimage}
  Here's one way: it suffices to prove that
  $d_A(\phi\circ\psi)(\phi\circ\psi)_*\mu=d_A(\phi)d_A(\psi)(\phi\circ\psi)_*\mu$
  (because there exists a compact set with positive finite measure)
  and using lemma~\ref{MeasureTheory.addEquivAddHaarChar_map}
  and the fact that $(\phi\circ\psi)_*\mu=\phi_*(\psi_*\mu)$
  one can simplify both sides to $\mu$.
\end{proof}

\subsection{Scaling Haar measure on a ring}

Now let $R$ be a locally compact topological ring. The \emph{Haar character} of $R$,
or more precisely the \emph{left Haar character} of $R$, is a group homomorphism
$R^\times\to\R^\times$ defined in the following way. If $u\in R^\times$ then left multiplication
by $u$, namely the map $\ell_u:(R,+)\to(R,+)$ defined by $\ell_u(r)=ur$, is a homeomorphism and
an additive automorphism of $(R,+)$, so the preceding theory applies to $\ell_u$.

\begin{definition}
  \label{MeasureTheory.ringHaarChar}
  \lean{MeasureTheory.ringHaarChar}
  \uses{MeasureTheory.addEquivAddHaarChar,MeasureTheory.addEquivAddHaarChar_refl,
  MeasureTheory.addEquivAddHaarChar_trans}
  \leanok
  We define $\delta_R(u)$ (or just $\delta(u)$ when the ring $R$ is clear) to be $d_R(\ell_u)$.
\end{definition}

Lemmas~\ref{MeasureTheory.addEquivAddHaarChar_refl}
and~\ref{MeasureTheory.addEquivAddHaarChar_trans} immediately imply that $\delta_R$ is a group
homomorphism from $R^\times$ to $\R_{>0}$.
Also immediate from previous lemmas is

\begin{lemma}
  \label{MeasureTheory.ringHaarChar_mul_integral}
  \lean{MeasureTheory.ringHaarChar_mul_integral}
  \uses{MeasureTheory.ringHaarChar}
  \leanok
  \discussion{514}
  If $f:R\to\R$ is a Borel measurable function and $u\in R^\times$ then
  $\delta_R(u)\int f(ux)d\mu(x)=\int f(x)d\mu(x)$.
\end{lemma}
\begin{proof}
  \uses{MeasureTheory.addEquivAddHaarChar_smul_integral_map}
  \leanok
  A short calculation using lemma~\ref{MeasureTheory.addEquivAddHaarChar_smul_integral_map}.
\end{proof}

\begin{lemma}
  \label{MeasureTheory.ringHaarChar_mul_volume}
  \lean{MeasureTheory.ringHaarChar_mul_volume}
  \uses{MeasureTheory.ringHaarChar,MeasureTheory.addEquivAddHaarChar_smul_preimage}
  \leanok
  \discussion{515}
  If $X$ is a Borel subset of $R$ and $r\in R^\times$ then $\mu(rX)=\delta_R(r)\mu(X)$.
\end{lemma}
\begin{proof}
  \leanok
  \uses{MeasureTheory.addEquivAddHaarChar_smul_preimage}
   Immediate from lemma~\ref{MeasureTheory.addEquivAddHaarChar_smul_preimage}.
\end{proof}

The next result lies a little deeper.

\begin{corollary}
  \label{MeasureTheory.ringHaarChar_continuous}
  \lean{MeasureTheory.ringHaarChar_continuous}
  \uses{MeasureTheory.ringHaarChar}
  \leanok
  \discussion{516}
  The function $\delta_R:R^\times\to\R_{>0}$ is continuous.
\end{corollary}
\begin{proof}
  \uses{MeasureTheory.ringHaarChar_mul_integral}
  Fix a Haar measure $\mu$ on $R$ and a continuous real-valued function
  $f$ on $R$ with compact support and such that $\int f(x) d\mu(x)\not=0$.
  Then $r \mapsto \int f(rx) d\mu(x)$ is a continuous function
  from $R\to\R$ (because a continuous function with compact support is uniformly
   continuous) and thus gives a continuous function $R^\times\to\R$.
   Thus the function $u\mapsto (\int f(ux) d\mu(x))/(\int f(x)d\mu(x))$ is
   a continuous function from $R^\times$ to $\R$ taking values in $\R_{>0}$.
   Hence $\delta_R^{-1}$ is continuous,
   from lemma~\ref{MeasureTheory.ringHaarChar_mul_integral},
   and thus $\delta_R$ is too.
\end{proof}

\section{Examples}

We discuss some examples of Haar characters.

\begin{lemma}
  \label{MeasureTheory.ringHaarChar_real}
  \lean{MeasureTheory.ringHaarChar_real}
  If $R=\R$ then $\delta_R(u)=|u|$.
  \leanok
\end{lemma}
\begin{proof}
  \uses{MeasureTheory.ringHaarChar_mul_volume}
  \leanok
  Take $\mu$ to be Lebesgue measure and $X=[0,1]$.
We have $\delta(u)=\mu(uX)$. If $u>0$ then $u[0,1]=[0,u]$ which has measure $u=|u|$,
and if $u<0$ then $u*[0,1]=[u,0]$ which has measure $-u=|u|$.
\end{proof}

\begin{lemma}
  \label{MeasureTheory.ringHaarChar_complex}
  \lean{MeasureTheory.ringHaarChar_complex}
  \leanok
  If $R=\bbC$ then $\delta_R(u)=|u|^2$.
\end{lemma}
\begin{proof}
  \uses{MeasureTheory.ringHaarChar_mul_volume}
  \leanok
  Multiplication by a positive real $r$ sends a unit square to a square of area $r^2=|r|^2$.
  Multiplication by $e^{i\theta}$ is a rotation and thus does not change area.
  The general case follows.
\end{proof}

\begin{lemma}
  \label{MeasureTheory.ringHaarChar_padic}
  \lean{MeasureTheory.ringHaarChar_padic}
  \leanok
  If $R=\Q_p$ then $\delta_R(u)=|u|_p$, the usual $p$-adic norm.
\end{lemma}
\begin{proof}
  \uses{MeasureTheory.ringHaarChar_mul_volume}
  \leanok
  Normalise Haar measure so that $\mu(\Z_p)=1$.
  If $u$ is a $p$-adic unit then $u\Z_p=\Z_p$ so multiplication by $u$ didn't change
  Haar measure. If however $u=p$ then $u\Z_p$ has index $p$ in $\Z_p$ and, because
  $\mu(i+p\Z_p)=\mu(p\Z_p)$ we have that $\mu(\Z_p)=p\mu(p\Z_p)$ and thus $\delta(p)=p^{-1}$.
  These elements generate $\Q_p^\times$ and two characters which agree on generators
  of a group must agree on the group.
\end{proof}

\begin{remark}
If $R$ is a finite extension of $\Q_p$ then $\delta_R(u)$
is the norm on $R$ normalised in the following way:
$\delta_R(\varpi)=q^{-1}$, where $\varpi$ is a uniformiser
and $q$ is the size of the (finite) residue field. In fact the same is true
for any nonarchimedean local field. The proof is
the same as for $\Q_p$. Right now this is difficult to state in Lean because
there is still some discussion about the definition of a nonarchimedean local field.
\end{remark}

\section{Algebras}

  Say $F$ is a locally compact topological field (for example $\R$ or $\bbC$ or $\Q_p$), $V$
  is a finite-dimensional $F$-vector space, and $\phi:V\to V$ is an invertible $F$-linear map.
  Then $V$ with its module topology (which is the product topology if one picks a basis)
  is a locally compact topological abelian group, and $\phi$ is additive.
  One can check that linearity implies continuity (this is {\tt IsModuleTopology.continuous\_of\_linearMap} in mathlib),
  so in fact $\phi$ is a homeomorphism
  and our theory applies. The following lemma gives a formula for the scale factor $d_V(\phi)$.

\begin{lemma}
  \label{MeasureTheory.addEquivAddHaarChar_eq_ringHaarChar_det}
  \lean{MeasureTheory.addEquivAddHaarChar_eq_ringHaarChar_det}
  \uses{MeasureTheory.ringHaarChar}
  \leanok
  \discussion{517}
  $d_V(\phi)=\delta_F(\det(\phi))$, where $\det(\phi)\in F$ is the determinant of $\phi$ as an $F$-linear map.
\end{lemma}
\begin{proof}
\uses{MeasureTheory.addEquivAddHaarChar}
The proof should be inspired by \href{https://leanprover-community.github.io/mathlib4\_docs/Mathlib/MeasureTheory/Measure/Lebesgue/Basic.html\#Real.map\_matrix\_volume\_pi\_eq\_smul\_volume\_pi}{\tt Real.map\_matrix\_volume\_pi\_eq\_smul\_volume\_pi},
which crucially uses the induction principle \href{https://leanprover-community.github.io/mathlib4\_docs/Mathlib/LinearAlgebra/Matrix/Transvection.html\#Matrix.diagonal\_transvection\_induction\_of\_det\_ne\_zero}{\tt Matrix.diagonal\_transvection\_induction\_of\_det\_ne\_zero}.
In short, one needs to check it for diagonal matrices and for matrices which are the identity
except that one off-diagonal entry is non-zero, because these matrices generate $GL_n(F)$
if $F$ is a field. Note that we only need the result for $F=\R$
and $F=\mathbb{Q}_p$ (and possibly for finite extensions of these depending on which route
we go for some intermediate results), so if it helps we can assume that $F$ is second countable
(but it shouldn't be necessary).
\end{proof}

Now say $F$ is still a locally compact topological field, and that $R$ is a (possibly
non-commutative) $F$-algebra. Recall that this means that ($R=0$ or) $F$ lies in the centre of $R$.
Assume that $R$ is finite-dimensional as an $F$-vector space. Then if we give $R$ the
$F$-module topology (which is just the product topology if we pick a basis) then it is known
that $R$ becomes a topological ring. Now say $u\in R^\times$, and
recall $\ell_u:R\to R$ is left multiplication by $u$. Then $\ell_u$ is easily checked to be
an $F$-linear homeomorphism.

  \begin{corollary}
  \label{MeasureTheory.algebra_ringHaarChar_eq_ringHaarChar_det}
  \lean{MeasureTheory.algebra_ringHaarChar_eq_ringHaarChar_det}
  \leanok
  If $u\in R^\times$ then $\delta_R(u)=\delta_F(\det(\ell_u))$.
\end{corollary}
\begin{proof}
  \uses{MeasureTheory.addEquivAddHaarChar_eq_ringHaarChar_det}
  \leanok
  Follows immediately from the preceding lemma.
\end{proof}

\section{Left and right multiplication}

If $R$ is a locally compact topological ring, and if multiplication on $R$ is not commutative,
then left and right multiplication by an element of~$R$ can scale Haar measure in different ways.
For example if $R$ is the upper-triangular $2\times 2$ matrices with real
entries, then left multiplication by $\begin{pmatrix}a&0\\0&1\end{pmatrix}$
sends $\begin{pmatrix}x&y\\0&z\end{pmatrix}$ to $\begin{pmatrix}ax&ay\\0&z\end{pmatrix}$
and thus scales $R$'s additive
Haar measure by $|a|^2$, but right multiplication by $\begin{pmatrix}a&0\\0&1\end{pmatrix}$
sends $\begin{pmatrix}x&y\\0&z\end{pmatrix}$ to $\begin{pmatrix}ax&y\\0&z\end{pmatrix}$
and thus scales $R$'s additive Haar measure by a factor of $|a|$.

What's going on here is that if we regard left and right multiplication as $\R$-linear
maps from $R$ to $R$, then their associated matrices with respect to the obvious basis
are $diag(a,a,1)$ and $diag(a,1,1)$, which have different determinants.

However, if $k$ is now any field and if $B$ is a finite-dimensional central
simple algebra over $k$ (for example a quaternion algebra, the case we'll care about later),
and if $u\in B^\times$ then $x\mapsto ux$ and $x\mapsto xu$
are both $k$-linear endomorphisms of $B$, and I claim that they have
the same determinant.

\begin{lemma}
  \label{IsSimpleRing.mulLeft_det_eq_mulRight_det}
  \lean{IsSimpleRing.mulLeft_det_eq_mulRight_det}
  \leanok
  \discussion{518}
  Say $B$ is a finite-dimensional central simple algebra over a field~$k$,
  and $u\in B^\times$. Let $\ell_u:B\to B$ be the $k$-linear maping $x$ to $ux$ and
  let $r_u:B\to B$ be the $k$-linear map sending $x$ to $xu$. Then
  $\det(\ell_u)=\det(r_u)$.
\end{lemma}
\begin{proof}
  Determinants are unchanged by base extension, so WLOG $k$ is algebraically closed.
  Then it's known that $B$ must be a matrix algebra, say $M_n(k)$. Now $u$ can be thought
  of as a matrix which has its own intrinsic determinant $d$, and $B$ as a left $B$-module
  becomes a direct sum of $n$ copies of $V$, the standard $n$-dimensional representation of $B$.
  Thus $\det(\ell_u)=d^n$. Similarly $\det(r_u)=d^n$ and in particular they are equal.
\end{proof}

\begin{corollary}
  \label{IsSimpleRing.ringHaarChar_eq_addEquivAddHaarChar_mulRight}
  \lean{IsSimpleRing.ringHaarChar_eq_addEquivAddHaarChar_mulRight}
  \leanok
  If $B$ is a central simple algebra over a locally compact field $F$, and if $u\in B^\times$,
  then $d_B(r_u)=\delta_B(u)$ (recall that the latter is defined to mean $d_B(\ell_u)$).
\end{corollary}
\begin{proof}
  \leanok
  \uses{IsSimpleRing.mulLeft_det_eq_mulRight_det, MeasureTheory.addEquivAddHaarChar_eq_ringHaarChar_det}
  If $\ell_u$ and $r_u$ denote left and right multiplication by $u$ on $B$, then we have
  seen in lemma~\ref{MeasureTheory.addEquivAddHaarChar_eq_ringHaarChar_det} that $d_B(r_u)=\delta_F(\det(r_u))$.
  Lemma~\ref{IsSimpleRing.mulLeft_det_eq_mulRight_det} tells
  us that this is $\delta_F(\det(\ell_u))$ and this is $\delta_B(u)$ again by
  corollary~\ref{MeasureTheory.addEquivAddHaarChar_eq_ringHaarChar_det}.
\end{proof}

\section{Finite Products}

Here are two facts which we will need about products.

\begin{lemma}
  \label{MeasureTheory.addEquivAddHaarChar_prodCongr}
  \lean{MeasureTheory.addEquivAddHaarChar_prodCongr}
  \leanok
  \discussion{520}
  If $(A,+)$ and $(B,+)$ are locally compact topological abelian groups,
  and if $\phi:A\to A$ and $\psi:B\to B$ are additive homeomorphisms,
  then $\phi\times\psi:A\times B\to A\times B$ is an additive homeomorphism (this is
  obvious), and
  $d_{A\times B}(\phi\times\psi)=d_A(\phi)d_B(\psi)$.
\end{lemma}
\begin{proof}
  We only need this result in the case where both $A$ and $B$ are second-countable, in which case
  {\tt Prod.borelSpace} can be used to show that Haar measure on $A\times B$ is the product of
  Haar measures on $A$ and $B$, and in this case the result follows easily. Without this assumption,
  the product of these measures may not even be a Borel measure and one has to be more careful.
  The proof in this case is explained by Gou\"{e}zel
  \href{https://leanprover.zulipchat.com/#narrow/channel/116395-maths/topic/Product.20of.20Borel.20spaces/near/487257981}{here}.
  Here is the idea. Let $\rho$ be a Haar measure on $A\times B$. Fix sets $X\subseteq A$ and
  $Y\subseteq B$ which are compact with nonempty interior. We can now pull back $\rho$
  to a measure $\nu$ on the Borel sigma-algebra of $A$ defined as $\nu(s)=\rho(s\times Y)$
  and this is easily checked to be a Haar measure on $A$. Then
  $\delta_{A\times B}(a,0)\nu(X)=
  \delta_{A\times B}(a,0)\rho(X\times Y)=\rho((a,0)(X\times Y))=
  \rho(aX\times Y)=\nu(aX)=\delta_A(a)\nu(X)$
  , so $\delta_{A\times B}(a,0)=\delta_A(a)$.
  Similarly $\delta_{A\times B}(0,b)=\delta_B(b)$ and because $\delta_{A\times B}$ is a group
  homomorphism we're home.
\end{proof}

\begin{lemma}
  \label{MeasureTheory.addEquivAddHaarChar_piCongrRight}
  \lean{MeasureTheory.addEquivAddHaarChar_piCongrRight}
  \leanok
  \discussion{521}
  \uses{MeasureTheory.addEquivAddHaarChar_prodCongr}
  If $A_i$ are a finite collection of locally compact topological abelian groups,
  with $\phi_i:A_i\to A_i$ additive homeomorphisms, then $d_{\prod_i A_i}(\prod_i\phi_i)=\prod_i d_{A_i}(\phi_i)$.
\end{lemma}
\begin{proof}
  Induction on the size of the finite set, using the previous lemma.
\end{proof}

\begin{lemma}
  \label{MeasureTheory.ringHaarChar_prod}
  \lean{MeasureTheory.ringHaarChar_prod}
  \leanok
  \uses{MeasureTheory.addEquivAddHaarChar_prodCongr}
  If $R$ and $S$ are locally compact topological rings, then $\delta_{R\times S}(r,s)=\delta_R(r)\times\delta_S(s)$.
\end{lemma}
\begin{proof}
  \leanok
  Follows immediately from lemma~\ref{MeasureTheory.addEquivAddHaarChar_prodCongr}.
\end{proof}

\begin{lemma}
  \label{MeasureTheory.ringHaarChar_pi}
  \lean{MeasureTheory.ringHaarChar_pi}
  \leanok
  \uses{MeasureTheory.addEquivAddHaarChar_piCongrRight}
  If $R_i$ are a finite collection of locally compact topological rings,
  and $u_i\in R_i^\times$ then $\delta_{\prod_i R_i}((u_i)_i)=\prod_i\delta_{R_i}(u_i)$.
\end{lemma}
\begin{proof}
  \leanok
  Follows immediately from lemma~\ref{MeasureTheory.addEquivAddHaarChar_piCongrRight}.
\end{proof}

\section{Some measure-theoretic preliminaries}

\begin{lemma}
  \label{Topology.IsOpenEmbedding.isHaarMeasure_comap}
  \lean{Topology.IsOpenEmbedding.isHaarMeasure_comap}
  \leanok
  \discussion{507}
  Let $A$ and $B$ be locally compact topological groups
  and let $f:A\to B$ be both a group homomorphism and open embedding.
  The pullback along $f$ of a Haar measure on $B$ is a Haar measure on $A$.
\end{lemma}
\begin{proof}
  \leanok
  Translation-invariance is easy, compact sets are finite because continuous
  image of compact is compact, open sets are bounded because image of open is open.
\end{proof}

\begin{lemma}
  \label{Topology.IsOpenEmbedding.regular_comap}
  \lean{Topology.IsOpenEmbedding.regular_comap}
  \leanok
  \discussion{513}
  The pullback of a regular Borel measure along an open
  embedding is a regular Borel measure.
\end{lemma}
\begin{proof}
  \leanok
  Again this is because the image of compact is compact and the
  image of open is open, so all the properties of being a regular measure
  are easily checked.
\end{proof}

\begin{lemma}
  \label{MeasureTheory.mulEquivHaarChar_eq_one_of_compactSpace}
  \lean{MeasureTheory.mulEquivHaarChar_eq_one_of_compactSpace}
  \leanok
  \discussion{532}
  Say $A$ is a compact topological additive group and $\phi:A\to A$ is an additive isomorphism.
Then $d_A(\phi)=1.$
\end{lemma}
\begin{proof}
  \uses{MeasureTheory.mulEquivHaarChar_smul_preimage}
  We have $d_A(\phi)\mu(A)=\mu(A)$
  from lemma~\ref{MeasureTheory.addEquivAddHaarChar_smul_preimage}
  and $\mu(A)$ is positive and finite because $A$ is open and compact.
\end{proof}

\begin{lemma}
  \label{MeasureTheory.addEquivAddHaarChar_eq_addEquivAddHaarChar_of_isOpenEmbedding}
  \lean{MeasureTheory.addEquivAddHaarChar_eq_addEquivAddHaarChar_of_isOpenEmbedding}
  \leanok
  \discussion{551}
  If $f:A\to B$ is a group homomorphism and open embedding between locally compact
  topological additive groups and if $\alpha:A\to A$ and $\beta:B\to B$ are additive
  homeomorphisms such that the square commutes (i.e., $f\circ\alpha=\beta\circ f$)
  then $d_A(\alpha)=d_B(\beta)$.
\end{lemma}
\begin{proof}
  \uses{Topology.IsOpenEmbedding.isHaarMeasure_comap,
    Topology.IsOpenEmbedding.regular_comap,
    MeasureTheory.addEquivAddHaarChar_smul_integral_comap}
  \leanok
  Choose a regular Haar measure $\mu_B$ on $B$. We just saw
  in lemmas~\ref{Topology.IsOpenEmbedding.isHaarMeasure_comap}
  and~\ref{Topology.IsOpenEmbedding.regular_comap} that its pullback
  $\mu_A:=f^*\mu_B$ to $A$ is also a regular Haar
  measure. Now fix a continuous compactly-supported function $g$ on $A$ with
  $0<\int g(a)d\mu(a)<\infty$. Then $d_A(\alpha)\int g(a)d\mu_A(a)=\int g(a)d(\alpha^*\mu_A)(a)$
  by lemma~\ref{MeasureTheory.addEquivAddHaarChar_smul_integral_comap}.
  This equals $\int g(a)d(\alpha^* f^*\mu_B)(a)$ by definition,
  which is $\int g(a)d(f^*\beta^*\mu_B)(a)$ because pullback of pullback is pullback.
  This equals $d_B(\beta)\int g(a) d(f^*\mu_B)(a)$ by
  corollary~\ref{MeasureTheory.addEquivAddHaarChar_comap}
  which is $d_B(\beta)\int g(a)d\mu_A(a)$ by definition,
  and so $d_A(\alpha)=d_B(\beta)$ as required.
\end{proof}

\section{Restricted products}

Now say $A=\prod'_i A_i$ is the restricted product of a collection of types $A_i$
  with respect to the subsets $C_i$. Recall that this is the subset of $\prod_i A_i$
  consisting of y Say $B=\prod'_i B_i$ is the restricted
  product of types $B_i$ over the same index set, with respect to
  subsets $D_i$. Say $\phi_i:A_i\to B_i$ are functions
  with the property that $\phi_i(C_i)\subseteq D_i$ for all but finitely many $i$.
It is easily checked that the $\phi_i$ induce a function $\phi:=\prod'_i\phi_i:A\to B$. It is also easily
checked that if all the $A_i$ and $B_i$ are groups or rings or $R$-modules, the $C_i$ and $D_i$
are subgroups or subrings or submodules,
and the $\phi_i$ are group or ring or module homomorphisms, then $\phi$ is a group or ring or
module homomorphism. However topological facts lie a little deeper.

\begin{lemma}
  \lean{Continuous.restrictedProduct_congrRight}
  \label{Continuous.restrictedProduct_congrRight}
  \leanok
  \discussion{531}
  If the $A_i$ and $B_i$ are topological spaces and the $\phi_i$ are continuous functions,
  then the restricted product $\phi = \prod'_i\phi_i$ is a continuous function.
\end{lemma}
\begin{proof}
  \leanok
  We use the universal property {\tt RestrictedProduct.continuous\_dom} of the
  topology in mathlib to reduce to the claim that for all finite $S$,
  the induced map $A_S:=\prod_{i\in S}A_i\times\prod_{i\notin S}C_i\to B$ is continuous.
  Because the inclusion $A_S\to A_T$ is continuous for $S\subseteq T$ we are reduced
  to checking this claim for $S$ sufficiently large that it contains all of the $i$
  for which $\phi(C_i)\not=D_i$. For such $S$, this map $A_S\to B$ factors as $A_S\to B_S\to B$
  and $B_S\to B$ is continuous, so it suffices to prove that $A_S\to B_S$ is continuous, but
  this is just a product of continuous maps.
\end{proof}

We now focus on the case that $B_i=A_i$ are locally compact groups, $D_i=C_i$ are compact
open subgroups, and $\phi_i:A_i\to A_i$ are group isomorphisms and homeomorphisms sending
$C_i$ to $C_i$ for all but finitely many $i$. Then the restricted product $A:=\prod'A_i$
of the $A_i$ with respect to the $C_i$ is also a locally compact topological group, and the
restricted product $\phi=\prod'\phi_i$ of the $\phi_i$ is a group isomorphism and homeomorphism,
so we can ask how $d_A(\phi)$ compares to the $d_{A_i}(\phi_i)$.

First note that $d_{A_i}(\phi_i)=1$ for all the $i$ such that $\phi_i(C_i)=C_i$, as
$d_{A_i}(\phi_i)$ can be computed as $\mu(\phi_i(C_i))/\mu(C_i)$ and $\mu(C_i)$ is
guaranteed to have positive finite measure as it is open and compact. Hence the product
$\prod_id_{A_i}\phi_i$ is a finite product, in the sense that all but finitely many terms are 1.
The following theorem shows that the value of this product is $d(\phi)$.

\begin{theorem}
  \label{MeasureTheory.addEquivAddHaarChar_restrictedProductCongrRight}
  \lean{MeasureTheory.addEquivAddHaarChar_restrictedProductCongrRight}
  \leanok
  \discussion{552}
  With $A$, $A_i$, $C_i$, $\phi_i$, $\phi$ defined as above, we have
  $\delta_A(\phi)=\prod_i\delta_{A_i}(\phi_i)$.
\end{theorem}
\begin{proof}
  \uses{MeasureTheory.addEquivAddHaarChar_eq_addEquivAddHaarChar_of_isOpenEmbedding,
  MeasureTheory.mulEquivHaarChar_eq_one_of_compactSpace,
  MeasureTheory.addEquivAddHaarChar_piCongrRight,Continuous.restrictedProduct_congrRight}
  Assume $\phi_i(C_i)=C_i$ for all $i\not\in S$, a finite set, and work in the
  open subgroup $U:=\prod_{i\in S}A_i\times\prod_{i\not\in S}C_i$. Then $\phi$ induces
  an automorphism $\phi_S$ of this open subgroup $U$ of $A$, and in particular
  lemma~\ref{MeasureTheory.addEquivAddHaarChar_eq_addEquivAddHaarChar_of_isOpenEmbedding} tells us
  that $\delta(\phi)=\delta_U(\phi_S)$.
  Next note that $\phi_S:U\to U$ can be written as a product of the automorphisms
  $\prod_{i\not\in S}\phi_i|_{C_i}$ of $\prod_{i\not\in S}C_i$ and
  $\prod_{i\in S}\phi_i$ of $\prod_{i\in S}A_i$. Because $\prod_{i\not\in S}C_i$ is a compact
  group we must have $\delta(\phi_{i\not\in S}\phi_i|_{C_i})=1$
  by lemma~\ref{MeasureTheory.mulEquivHaarChar_eq_one_of_compactSpace}. Finally
  $\delta(\prod_{i\in S}\phi_i)=\prod_{i\in S}\delta(\phi_i)$ by
  lemma~\ref{MeasureTheory.addEquivAddHaarChar_piCongrRight} and we are home.
\end{proof}

As a special case, if $R$ is the restricted product of a collection of topological rings $R_i$
  (not necessarily commutative) each equipped with a compact open subring $C_i$, then
  we have

\begin{corollary}
  \label{MeasureTheory.ringHaarChar_restrictedProduct}
  \lean{MeasureTheory.ringHaarChar_restrictedProduct}
  \leanok
  \discussion{554}
  If $u=(u_i)_i\in R^\times$ then $\delta_R(u)=\prod_i\delta_{R_i}(u_i)$.
\end{corollary}
\begin{proof}
  \leanok
  \uses{MeasureTheory.addEquivAddHaarChar_restrictedProductCongrRight}
  By definition of restricted product we have $u_i\in C_i$ for all but finitely many $i$.
  Note also that $u$ has an inverse $v=(v_i)_i$ with $v_i\in C_i$ for all but finitely many $i$.
  The fact that $u_iv_i=v_iu_i=1$ means that $u_i,v_i\in C_i^\times$ for all but finitely many $i$.
  Thus the previous theorem~\ref{MeasureTheory.addEquivAddHaarChar_restrictedProductCongrRight} applies.
\end{proof}

\section{Adeles}

We finish this miniproject by proving some results about Haar characters for
algebras over adele rings.
So let $K$ be a number field and let $\A_K$ be the adeles of $K$.
We will prove some theorems about $\A_K$-algebras $R$ which are finite
and free as $\A_K$-modules. Such algebras can be given the $\A_K$-module topology
and this makes them into locally compact topological rings. In fact we shall only be concerned
in applications with algebras of the form $B\otimes_K\A_K$ where $B$ is a finite-dimensional
$K$-algebra. So fix such a $B$, and write $B_{\A}$ for $B\otimes_K\A_K$. Let us first
deal with a subtlety. Recall that if $K\subseteq L$ are number fields, then $\A_L$ is a
module-finite $\A_K$-algebra and hence an $\A_K$-module, and
theorem~\ref{NumberField.AdeleRing.baseChange_moduleTopology}
tells us that $\A_L$ has the $\A_K$-module topology. Thus the next lemma applies.

\begin{lemma}
  \label{IsModuleTopology.Module.continuous_bilinear_of_finite}
  \lean{IsModuleTopology.Module.continuous_bilinear_of_finite}
  \leanok
  Say $R$ and $S$ are topological rings, and $S$ is an $R$-algebra, finite as an $R$-module.
  Assume that the topology
  on $S$ is the $R$-module topology. Now say $M$ is an $S$-module, and give it the induced
  $R$-module structure. Then the $R$-module topology and $S$-module topology on~$M$ coincide.
\end{lemma}
\begin{proof}
  \leanok
  Let $i:R\to S$ denote the structure map. First observe that $S$ has the $R$-module topology
  so the $R$-action map $R\times S\to S$ (explicitly defined by $(r,s)\mapsto i(r)s$)
  is continuous, and restricting to $s=1$ we deduce that $i$ is continuous.

  Now let $M_R$ and $M_S$ denote $M$ with the $R$-module and $S$-module topologies respectively.
  It suffices to prove that the identity maps $M_R\to M_S$ and $M_S\to M_R$ are continuous.
  Equivalently, because the $A$-module topology on an $A$-module is the finest topology
  making it into a topological module, we need to prove that $M_R$ is a topological $S$-module
  and that $M_S$ is a topological $R$-module. We start with the latter claim.

  First observe that $M_S$ is a topological $S$-module, so addition is continuous.
  Next note that the map $R\times M_S\to M_S$ factors through $S\times M_S$ and is hence the
  composite of two continuous maps and thus continuous. Hence $M_S$ is a topological $R$-module.

  It thus remains to check that $M_R$ is a topological $S$-module, or equivalently
  that the map $S\times M_R\to M_R$ is continuous. But this map is $R$-bilinear, and
  by the result {\tt Module.continuous\_bilinear\_of\_finite} in {\tt mathlib}, any
  $R$-bilinear map between modules with the $R$-module topology is automatically continuous
  if one of the source modules is finitely-generated. The result applies because $S$ is
  assumed to be a finite $R$-module and the proof is complete.
\end{proof}

\begin{corollary}
  \label{NumberField.AdeleRing.ModuleBaseChangeContinuousAddEquiv}
  \lean{NumberField.AdeleRing.ModuleBaseChangeContinuousAddEquiv}
  \leanok
  If $K$ is a number field and $V$ is an $K$-module, then
  the natural isomorphism $V\otimes_K\A_K=V\otimes_{\Q}\A_{\Q}$ induced by the natural
  isomorphism $\A_K=K\otimes_K\A_{\Q}$ is a homeomorphism if the left hand side has the $\A_K$-module
  topology and the right hand side has the $\A_{\Q}$-module topology.
\end{corollary}
\begin{proof}
  \uses{IsModuleTopology.Module.continuous_bilinear_of_finite}
  Lemma~\ref{IsModuleTopology.Module.continuous_bilinear_of_finite} tells us that $V\otimes_K\A_K$
  has the $\A_{\Q}$-module topology, and it is easily checked that the isomorphism is
  $\A_{\Q}$-linear and hence automatically continuous.

  Note that in the Lean we prove this for a general extension $L/K$ rather than $K/\Q$.
\end{proof}

As a consequence, if $B$ is a $K$-algebra then we can think of $B_{\A}$ as either $B\otimes_K\A_K$
with the $\A_K$-module topology or as $B\otimes_{\Q}\A_{\Q}$ with the $\A_{\Q}$-module
topology. Note that this isomorphism commutes with the inclusions from $B$ into these rings.
But Lean is picky about these things so we'll have to be careful.

\begin{theorem}
  \label{NumberField.AdeleRing.isCentralSimple_addHaarScalarFactor_left_mul_eq_right_mul}
  \lean{NumberField.AdeleRing.isCentralSimple_addHaarScalarFactor_left_mul_eq_right_mul}
  \leanok
  Let $B$ be a finite-dimensional central simple $K$-algebra.
  Say $u\in B_{\A}^\times$, and define $\ell_u$ and $r_u:B_{\A}\to B_{\A}$ by
  $\ell_u(x)=ux$ and $r_u(x)=xu$. Then $d_{B_{\A}}(\ell_u)=d_{B_{\A}}(r_u)$.
\end{theorem}
\begin{proof}
  \uses{MeasureTheory.addEquivAddHaarChar_restrictedProductCongrRight,
    IsSimpleRing.ringHaarChar_eq_addEquivAddHaarChar_mulRight}
  We think of $B_{\A}$ as $B\otimes_K\A_K$.
  If $u=(u_v)$ as $v$ runs through the places of $K$ then
  $d_{B_{\A}}(\ell_u)=\prod_v d_{B_v}(\ell_{u_v})$ by
  theorem~\ref{MeasureTheory.addEquivAddHaarChar_restrictedProductCongrRight} (and the product is finite).
  By corollary~\ref{IsSimpleRing.ringHaarChar_eq_addEquivAddHaarChar_mulRight}
  this equals $\prod_v d_{B_v}(r_{u_v})$, and again by
  theorem~\ref{MeasureTheory.addEquivAddHaarChar_restrictedProductCongrRight} this is $d_{B_{\A}}(r_u)$.
\end{proof}

The previous theorem only applies to inner forms of matrix algebras, but the below theorem,
a generalization of the adelic product formula, is valid for any finite-dimensional
$K$-algebra. Before we state it let's remind ourselves of the product formula for $\Q$,
and restate it in the language of these Haar characters.

\begin{lemma}
  \label{MeasureTheory.ringHaarChar_adeles_rat}
  \lean{MeasureTheory.ringHaarChar_adeles_rat}
  \leanok
  If $x\in\A_{\Q}^\times$ then $\delta_{\A_{\Q}}(x)=\prod_v|x_v|_v.$
\end{lemma}
\begin{proof}
  \uses{MeasureTheory.addEquivAddHaarChar_prodCongr,MeasureTheory.ringHaarChar_real,
    MeasureTheory.addEquivAddHaarChar_restrictedProductCongrRight,MeasureTheory.ringHaarChar_padic}
  By theorem~\ref{MeasureTheory.addEquivAddHaarChar_prodCongr}
  we have $\delta_{\A_{\Q}}(x)=\delta_{\A_{\Q}^\infty}(x^\infty)\times\delta_{\R}(x_\infty)$.
  By lemma~\ref{MeasureTheory.ringHaarChar_real} we have $\delta_{\R}(x_\infty)=|x|_\infty$, and by
  theorem~\ref{MeasureTheory.addEquivAddHaarChar_restrictedProductCongrRight} we have
  $\delta_{\A_{\Q}^\infty}=\prod_p\delta_{\Q_p}(x_p)$. By lemma~\ref{MeasureTheory.ringHaarChar_padic}
  we have $\delta_{\Q_p}(x_p)=|x_p|_p$ and putting everything together we get the result.
\end{proof}

Now $\A_{\Q}$ is nonzero a $\Q$-algebra and hence we have an inclusion $\Q^\times\to\A_{\Q}^\times$.
Here is our reinterpretation of the product formula.

\begin{lemma}
  \label{MeasureTheory.ringHaarChar_adeles_units_rat_eq_one}
  \lean{MeasureTheory.ringHaarChar_adeles_units_rat_eq_one}
  \leanok
  If $x\in\Q^\times\subseteq\A_{\Q}^\times$ then $\delta_{\A_{\Q}}(x)=1.$
\end{lemma}
\begin{proof}
  \uses{MeasureTheory.ringHaarChar_adeles_rat}
  By lemma~\ref{MeasureTheory.ringHaarChar_adeles_rat} we have $\delta_{\A_{\Q}}(x)=\prod_v|x|_v$.
  But the product formula says that this is 1.
  A quick proof: if $x=\pm\prod_pp^{e_p}$ then $\prod_p|x|_p=\prod_pp^{-e_p}$
  and $|x|_\infty=\prod_pp^{e_p}$ so they cancel.
\end{proof}

  Next we generalize this to finite-dimensional $\Q$-vector spaces.

  So say $V$ is an $N$-dimensional $\Q$-vector space,
  and define $V_{\A}:= V\otimes_{\Q}\A_{\Q}$ with its $\A_{\Q}$-module topology.
  If we choose an isomorphism $V\cong\Q^N$ then $V_{\A}\cong\A_{\Q}^N$
  as an additive topological abelian group. In particular, $V_{\A}$ is locally compact.

  Fix a $\Q$-linear automorphism $\phi:V\to V$. By base extension $\phi$ induces
  an $\A_{\Q}$-linear automorphism $\phi_{\A}$ of $V_{\A}$ which is also a homeomorphism of $V_{\A}$
  if $V_{\A}$ is given the module topology as an $\A_{\Q}$-module. Our goal is

  \begin{theorem}
    \label{MeasureTheory.addHaarScalarFactor_tensor_adeles_eq_one}
    \lean{MeasureTheory.addHaarScalarFactor_tensor_adeles_eq_one}
    \leanok
    In the above situation ($V$ a finite-dimensional $\Q$-vector space, $\phi:V\cong V$ is
    $\Q$-linear, $\phi_{\A}$ the base extension to $V_{\A}:=V\otimes_{\Q}{\A_{\Q}}$, a continuous linear
    endomorphism of $V_{\A}$ with the $\A_{\Q}$-module topology), we have $d_{V_{\A}}(\phi_{\A})=1.$
  \end{theorem}
  \begin{proof}
    \uses{
      MeasureTheory.addEquivAddHaarChar_eq_ringHaarChar_det,
      MeasureTheory.addEquivAddHaarChar_restrictedProductCongrRight,
      NumberField.AdeleRing.ModuleBaseChangeContinuousAddEquiv
    }
    Fix once and for all a $\Z$-lattice $L\subseteq V$
    (that is, a spanning $\Z^N$ in the $\Q^N$). Then $V\otimes_{\Q}\A_{\Q}=L\otimes_{\Z}\A_{\Q}$
    which is $L\otimes_{\Z}(\A_{\Q}^\infty\times\R)$.
    Because tensoring by a finitely presented module commute with restricted products and binary
    products this equals $\prod'_p(L\otimes_{\Z}\Q_p)\times(L\otimes_{\Q}\R)$, the restricted
    product being over $L\otimes_{\Z}\Z_p$, and this is $\prod'_p(V\otimes_{\Q}\Q_p)\times(V\otimes_{\Q}\R)$.
    If $V_v$ denotes $V\otimes_{\Q}\Q_v$ then the endomorphism $\phi_{\A}$ is the product of
    $\phi_p$ for all $p$ and $\phi_\infty$,
    where $\phi_v$ is a $\Q_v$-linear and and hence continuous automorphism of $V_v$.

    Hence by theorem~\ref{MeasureTheory.addEquivAddHaarChar_restrictedProductCongrRight}
    we have $d_{V_{\A}}(\phi_{\A})=\prod_p d_{V_p}(\phi_p)\times d_{V_\infty}(\phi_\infty)$,
    where we note that all but finitely many of the $d_{V_p}(\phi_p)$ are 1.

    By Lemma~\ref{MeasureTheory.addEquivAddHaarChar_eq_ringHaarChar_det} we have that
    $d_{V_v}(\phi_v)=\delta_{\Q_v}(\det(\phi_v))$, hence $d_{V_{\A}}(\phi_{\A})=\prod_v\delta_{\Q_v}(\det(\phi_v))$.
    But $\det(\phi_v)$ is equal to the determinant of $\phi$ on $V$ as $\Q$-vector space (because
    base change does not change determinant),
    which is some nonzero rational number $q$. Thus $d_{V_{\A}}(\phi_{\A})=\prod_v\delta_{\Q_v}(q)=1$
    by the product formula for $\Q$.
  \end{proof}

  \begin{corollary}
    \label{NumberField.AdeleRing.units_mem_ringHaarCharacter_ker}
    \lean{NumberField.AdeleRing.units_mem_ringHaarCharacter_ker}
    \leanok
    If $B$ is a finite-dimensional $\Q$-algebra (for example a number field, or a quaternion algebra over a number field),
    if $B_{\A}$ denotes the ring $B\otimes_{\Q}\A_{\Q}$, and if $b\in B^\times$,
    then $\delta_{B_{\A}}(b)=1$.
  \end{corollary}
  \begin{proof}
    \uses{MeasureTheory.addHaarScalarFactor_tensor_adeles_eq_one}
    Follows immediately from the previous theorem.
  \end{proof}

  \begin{corollary}
    \label{NumberField.AdeleRing.addEquivAddHaarChar_mulRight_unit_eq_one}
    \lean{NumberField.AdeleRing.addEquivAddHaarChar_mulRight_unit_eq_one}
    \leanok
    If $B$ is a finite-dimensional $\Q$-algebra and
    if $b\in B^\times$ then right multiplication by $b$
    does not change Haar measure on $B_{\A}$.
  \end{corollary}
  \begin{proof}
    \uses{MeasureTheory.addHaarScalarFactor_tensor_adeles_eq_one}
    Follows immediately from the previous theorem.
  \end{proof}
